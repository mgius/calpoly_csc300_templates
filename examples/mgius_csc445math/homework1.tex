\documentclass[11pt]{article}
% decent example of doing mathematics and proofs in LaTeX.
% An Incredible degree of information can be found at
% http://en.wikibooks.org/wiki/LaTeX/Mathematics

% Use wide margins, but not quite so wide as fullpage.sty
\marginparwidth 0.5in 
\oddsidemargin 0.25in 
\evensidemargin 0.25in 
\marginparsep 0.25in
\topmargin 0.25in 
\textwidth 6in \textheight 8 in
% That's about enough definitions

\usepackage{amsmath}
\usepackage{upgreek}

\begin{document}
\author{Mark Gius}
\title{Homework Set 1: 141 Redux}
\maketitle

\begin{enumerate}

\item % Problem 1: 

Prove that:
% starts math environment, multiline
\[
1^2 + 2^2 + \cdots + n^2 = \frac{n(n+1)(2n+1)}{6}
\]

Using Proof by Induction.

First, prove that for some n, this equation holds true.

% starts math environment with alignment on a particular pivot.  The pivot is
% denoted by '&' on each line
\begin{align*}
n = 2
1^2 + 2^2 & = \frac{2(2+1)(4+1)}{6} \\
2 + 4 & = \frac{2(3)(5)}{6} \\
6 &= \frac{30}{6} \\
6 &= 6
\end{align*}

Now, prove that this works for any n+1.

\begin{align*}
1^2 + 2^2 + \cdots + n^2 & = \frac{n(n+1)(2n+1)}{6} \\
1^2 + 2^2 + \cdots + n^2 + (n+1)^ 2 & = \frac{(n+1)(n+2)(2n+3)}{6} \\
\intertext{Notice that the n+1 equation contains the n equation.}
\boxed{1^2 + 2^2 + \cdots + n^2 } + (n+1)^ 2 & = \frac{(n+1)(n+2)(2n+3)}{6} \\
\frac{n(n+1)(2n+1)}{6} + (n+1)^2 & = \frac{(n+1)(n+2)(2n+3)}{6} \\
n(n+1)(2n+1) + 6(n+1)^2 & = (n+1)(n+2)(2n+3) \\
n(2n+1) + 6(n+1) & = (n+2)(2n+3) \\
2n^2 + n + 6n + 6 & = 2n^2 + 4n + 3n + 6 \\
2n^2 + 7n + 6 &= 2n^2 + 7n + 6 \\
\intertext{therefore}
1^2 + 2^2 + \cdots + n^2 & = \frac{n(n+1)(2n+1)}{6}
\end{align*}

\item % Problem 2
Prove that

\[
6 \mid n^3 - n
\]

Using Proof by Induction.

First prove that this equation is valid for an arbitrary n.

\begin{align*}
n & = 2 \\
6 & \mid 2^3 - 2 \\
6 & \mid 6
\end{align*}

Now, prove for any n+1

\begin{align*}
6 & \mid (n+1)^3 - (n+1) \\
6 & \mid n^3 + 3n^2 + 3n + 1 - n - 1 \\
6 & \mid n^3 + 3n^2 + 3n - n \\
\intertext{I can pull the original equation out of this one}
6 & \mid \boxed{n^3 - n} + 3n^2 + 3n \\
\intertext{Now I need to prove that $3n^2 + 3n$ is divisible by 6}
6 & \mid 3n^2 + 3n \\
6 & \mid 3(n^2 + n) \\
2 & \mid n^2 + n \\ 
\end{align*}
% this line contains a snippet of math.  Rather than owning its own line, this 
% math equation is integrated in into the surrounding text.
Now I need to prove that $n^2 + n$ is divisible by 2, or even 

Let n be even. By our proof in class today (seen in one form in problem 3),
$n^2$ is even when $n$ is even.  An even number added to an even number is even.

Let n be odd. By the same proof, $n^2$ is odd when $n$ is odd. An odd number
added to an odd number is an even number. Therefore, $n^2 + n$ is an even
number. 

Therefore, $6 \mid (n+1)^3 - (n+1)$.

\item % Problem 3
Prove that $\sqrt[3]{2}$ is an irrational number

Assume that $\sqrt[3]{2}$ is a rational number.  If so, then
\begin{align*}
\sqrt[3]{2} = \frac{a}{b} & != 0 \\
\intertext{\em{where a,b have no common factors}}
% this command embeds normal text in a math context.  Very useful.
a^3 & = b\sqrt[3]{2} \\
a^3 & = 2b^3 
\intertext{We now know that $a^3$ is even. It would be helpful is a was even.}
\intertext{Let $a^3$ be even, prove that a is even}
a^3 & = 2k \\
a & = \sqrt[3]{2k} \\
a & = \sqrt[3]{2} \sqrt[3]{k} \\
\intertext{Blech...lets try again with the contrapositive.}
% and here I'm embedding normal text with a math snippet in it.
\intertext{Assume that a is odd, prove $a^3$ is odd.}
a & = 2k + 1 \\
a^3 & = 8k^3 + 12k^2 + 6k + 1 \\
a^3 & = 2(4k^3 + 6k^2 + 3k) + 1 \\
\intertext{$a^3$ is odd, therefore if $a^3$ is even, a is even.  Now back.}
a^3 & = 2b^3 \\
(2L)^3 & = 2b^3 \\
8L^3 & = 2b^3 \\
4L^3 & = b^3 \\
2(2L^3) & = b^3 \\
\intertext{By the same proof as above, $b$ must be even because $b^3$ is even.
Now, a and b share the common factor of two, therefore, $\sqrt[3]{2}$ is not
rational, and therefore irrational.}
\end{align*}
\item % Problem 4
Given G(V,E), we know that $\sum_{1}^{n} \mid d_i = 2e$. Prove
\[
e \leq \frac{n(n-1)}{2}
\]

\begin{align*}
\intertext{Base case $n=2$, a graph with two vertices has 1 edge.}
e & = 1 \\
e & \leq \frac{2(2-1)}{2} \\
1 & \leq 1 \\
\intertext{Now prove the $n+1$ option. When the $n+1$ vertice is added, it can
add up to $n$ edges, one for each of the existing vertices.}
e + n & \leq \frac{(n+1)n}{2} \\
e + n & \leq \frac{n^2 + n}{2} \\
e + n & \leq \frac{n^2 + n + n - n}{2} \\
e + n & \leq \frac{n^2 - n}{2} + \frac{2n}{2} \\
e + n & \leq \boxed{\frac{n(n-1)}{2}} + n \\
\intertext{Therefore, by induction:}
e & \leq \frac{n(n-1)}{2}
\end{align*}

\item % Problem 5

Show that every graph with two or more nodes contains two nodes that have equal
degrees.

Let us try to prove that every graph with two or more nodes have unique
degrees.  We know that the set of possible degrees for a graph with $n$ vertices
is:

\[
0,1,\ldots,n-1
\]

This gives us a total of $n$ unique degrees to assign to our $n$ vertices. We
must assign a degree of zero to one vertex. A vertex with degree zero is
connected to no other vertices.  Let us now assign the degree $n-1$ to a
vertice.  This vertice is connected to every other vertice in the graph.
This is a contradiction, because it is impossible to simulatenously have a
vertice that is connected to every other vertice, and a vertice that is
connected to none.  Therefore, there are at least two vertices with the same
degree in any graph with at least 2 vertices.
\end{enumerate}
\end{document}
